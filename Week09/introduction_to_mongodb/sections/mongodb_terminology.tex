\section{MongoDB terminology}\label{sec:mongodb_terminology}
Before we dive into MongoDB it is good to run through the terminology used in MongoDB. If you don't grasp these terms right away -- no worries. Read through and they will be explained in examples below. The revisit this section and read through them again.

A record in MongoDB is a document, which is a data structure composed of field and value pairs. MongoDB documents are similar to JSON objects. The values of fields may include other documents, arrays, and arrays of documents.

MongoDB stores documents in collections. Collections are analogous to tables in relational databases. Unlike a table, however, a collection does not require its documents to have the same schema.

MongoDB stores collections within a Database. You can have multiple databases within MongoDB. A database should group together collections semantically by applications.
