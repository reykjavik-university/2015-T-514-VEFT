\section{Starting up MongoDB locally}\label{sec:stargin_up_mongodb_locally}
To start up a single instance node of MongoDB locally we use the \texttt{mongod} command.

\texttt{mongod} is a highly configurable service, as \texttt{mongod --help} indicates. But for this tutorial we will execute it with default configuration. The only configuration parameter that we need to give is the \texttt{--dbpath} flag. With this flag we tell \texttt{mongod} where to store the database data and meta-data.

In this document we use the folder \texttt{/tmp/data}. We need to create this folder before we start up \texttt{mongod}

\begin{lstlisting}
mkdir /tmp/data
\end{lstlisting}

The only requirement for this folder is that the \texttt{mongod} process has read and write access to it. You can select any folder that you want, but we use the above mentioned folder in the text.

Now to start a single node run \texttt{mongod} as follows.

\begin{lstlisting}
mongod --dbpath /tmp/data
\end{lstlisting}

You should now have a local MongoDB node running on your machine listening to port $27019$.
