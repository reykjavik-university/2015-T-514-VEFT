\section{Installing MongoDB}\label{sec:installing_mongodb}
In this section we go through the steps needed for installing MongoDB server locally. MongoDB's documentation website\footnote{\url{http://docs.mongodb.org/}} includes great help on that matter. If these instructions do not work for your operating system then we suggest that you take a look at that documentation instead.

After the install you should have at least two MongoDB commands available in your shell, \texttt{mongo} and \texttt{mongod}. After the install verify that these commands are available in your shell.

\subsection{Ubuntu Linux}
For Ubuntu we can use APT (Advanced Package Tool) to install MongoDB.Open up a terminal and type in the following lines.

\begin{lstlisting}
sudo apt-get update
sudo apt-get install mongodb
sudo update-rc.d -f mongodb remove
sudo service mongodb stop
\end{lstlisting}

The last two lines disable MongoDB to start on a boot. If you want to start MongoDB on boot, skip the last two lines.

\subsection{OSX}
We can use Homebrew\footnote{\url{http://brew.sh/}} to install MongoDB as follows.

\begin{lstlisting}
brew install mongodb
\end{lstlisting}

We can also download pre-built binaries and add them to your PATH environment variable.
\begin{lstlisting}
cd
wget https://fastdl.mongodb.org/osx/
     mongodb-osx-x86_64-3.0.6.tgz
tar xzvf mongodb-osx-x86_64-3.0.6.tgz
\end{lstlisting}

Then add the bin folder within the mongodb-osx-x86\_64-3.0.6 to your PATH variable as follows.

\begin{lstlisting}
export PATH="$HOME/mongodb-osx-x86_64-3.0.6/bin:$PATH"
\end{lstlisting}

You can add this line to your shell configuration.
