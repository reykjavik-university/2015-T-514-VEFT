\section{Array}\label{sec:arrays}
In this section we introduce the Array object in JavaScript.

\subsection{Create array}
To create an empty array in JavaScript we use \texttt{[]} as the following example indicates

\begin{lstlisting}
> const a = [];
undefined
> a
[]
\end{lstlisting}

We can also initialize the array with values by placing them between the brackets.

\begin{lstlisting}
> const a = [1, 2, 3, 4];
undefined
> a
[ 1, 2, 3, 4 ]
\end{lstlisting}

In JavaScript you can mix objects of different types into the array. They are not bound to any type.

\begin{lstlisting}
> [1, 'hlysig', '2', {}]
[ 1, 'hlysig', '2', {} ]
\end{lstlisting}

\subsection{Array operations}
Arrays in JavaScript are 0-based. That is, the first value in the array has the index zero. We can fetch values from they array by the index as follows.

\begin{lstlisting}
> const a = ['hlysig', 'f00', 'yeah']
undefined
> a[1]
'f00'
\end{lstlisting}

If you select an out-of-bounds index, then you will get \texttt{undefined} back.

The Array object has many functions to to operate over an array instance. There is an exceilent article on that on the MDN\footnote{\url{https://developer.mozilla.org/en-US/docs/Web/JavaScript/Reference/Global_Objects/Array}}. But in the following listings we will show a hand fill of them.

The \texttt{slice} method returns a shallow copy of a portion of an array into a new array object.
\begin{lstlisting}
> a
[ 'hlysig', 'f00', 'yeah' ]
> a.slice(-1)
[ 'yeah' ]
> a.slice(0, 1)
[ 'hlysig' ]
\end{lstlisting}

The \texttt{length} property returns the length of the array.
\begin{lstlisting}
> a.length
3
\end{lstlisting}

The \texttt{indexOf} returns the index of a given value within the array.

\begin{lstlisting}
> a.indexOf('f00')
1
\end{lstlisting}

The \texttt{map} function receives a function and maps that function over each element of the array and returns a new array.

\begin{lstlisting}
> const d = [1, 2, 3, 4];
undefined
> d.map((v) => v+2)
[ 3, 4, 5, 6 ]
\end{lstlisting}

The \texttt{forEach} function iterates over the whole array. This function accepts a call-back function which is called for each element in the array

\begin{lstlisting}
> d.forEach(function(v) {
    console.log(v)
})
1
2
3
4
\end{lstlisting}

When you need some specific array operation, please look at the MDN article on Arrays. You will most likely find it there.

